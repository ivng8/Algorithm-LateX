\documentclass[11pt]{article}

\newcommand{\yourname}{Ivan Ng}
\newcommand{\yourcollaborators}{Marcus Yi}

\def\comments{0}

%format and packages

%\usepackage{algorithm, algorithmic}

\usepackage{ulem}
\usepackage{epsfig, graphicx}
\usepackage[noend]{algpseudocode}
\usepackage{amsmath, amssymb, amsthm}
\usepackage{enumerate}
\usepackage{enumitem}
\usepackage{framed}
\usepackage{verbatim}
\usepackage[margin=1.1in]{geometry}
\usepackage{microtype}
\usepackage{kpfonts}
\usepackage{palatino}
	\DeclareMathAlphabet{\mathtt}{OT1}{cmtt}{m}{n}
	\SetMathAlphabet{\mathtt}{bold}{OT1}{cmtt}{bx}{n}
	\DeclareMathAlphabet{\mathsf}{OT1}{cmss}{m}{n}
	\SetMathAlphabet{\mathsf}{bold}{OT1}{cmss}{bx}{n}
	\renewcommand*\ttdefault{cmtt}
	\renewcommand*\sfdefault{cmss}
	\renewcommand{\baselinestretch}{1.05}
\usepackage[usenames,dvipsnames]{xcolor}
\definecolor{DarkGreen}{rgb}{0.15,0.5,0.15}
\definecolor{DarkRed}{rgb}{0.6,0.2,0.2}
\definecolor{DarkBlue}{rgb}{0.2,0.2,0.6}
\definecolor{DarkPurple}{rgb}{0.4,0.2,0.4}
\usepackage[pdftex]{hyperref}
\hypersetup{
	linktocpage=true,
	colorlinks=true,				% false: boxed links; true: colored links
	linkcolor=DarkBlue,		% color of internal links
	citecolor=DarkBlue,	% color of links to bibliography
	urlcolor=DarkBlue,		% color of external links
}

\usepackage[boxruled,vlined,nofillcomment]{algorithm2e}
	\SetKwProg{Fn}{Function}{\string:}{}
	\SetKwFor{While}{While}{}{}
	\SetKwFor{For}{For}{}{}
	\SetKwIF{If}{ElseIf}{Else}{If}{:}{ElseIf}{Else}{:}
	\SetKw{Return}{Return}
	

%enclosure macros
\newcommand{\paren}[1]{\ensuremath{\left( {#1} \right)}}
\newcommand{\bracket}[1]{\ensuremath{\left\{ {#1} \right\}}}
\renewcommand{\sb}[1]{\ensuremath{\left[ {#1} \right\]}}
\newcommand{\ab}[1]{\ensuremath{\left\langle {#1} \right\rangle}}

%probability macros
\newcommand{\ex}[2]{{\ifx&#1& \mathbb{E} \else \underset{#1}{\mathbb{E}} \fi \left[#2\right]}}
\newcommand{\pr}[2]{{\ifx&#1& \mathbb{P} \else \underset{#1}{\mathbb{P}} \fi \left[#2\right]}}
\newcommand{\var}[2]{{\ifx&#1& \mathrm{Var} \else \underset{#1}{\mathrm{Var}} \fi \left[#2\right]}}

%useful CS macros
\newcommand{\poly}{\mathrm{poly}}
\newcommand{\polylog}{\mathrm{polylog}}
\newcommand{\zo}{\{0,1\}}
\newcommand{\pmo}{\{\pm1\}}
\newcommand{\getsr}{\gets_{\mbox{\tiny R}}}
\newcommand{\card}[1]{\left| #1 \right|}
\newcommand{\set}[1]{\left\{#1\right\}}
\newcommand{\negl}{\mathrm{negl}}
\newcommand{\eps}{\varepsilon}
\DeclareMathOperator*{\argmin}{arg\,min}
\DeclareMathOperator*{\argmax}{arg\,max}
\newcommand{\eqand}{\qquad \textrm{and} \qquad}
\newcommand{\ind}[1]{\mathbb{I}\{#1\}}
\newcommand{\sslash}{\ensuremath{\mathbin{/\mkern-3mu/}}}

%info theory macros
\newcommand{\SD}{\mathit{SD}}
\newcommand{\sd}[2]{\SD\left( #1 , #2 \right)}
\newcommand{\KL}{\mathit{KL}}
\newcommand{\kl}[2]{\KL\left(#1 \| #2 \right)}
\newcommand{\CS}{\ensuremath{\chi^2}}
\newcommand{\cs}[2]{\CS\left(#1 \| #2 \right)}
\newcommand{\MI}{\mathit{I}}
\newcommand{\mi}[2]{\MI\left(~#1~;~#2~\right)}

%mathbb
\newcommand{\N}{\mathbb{N}}
\newcommand{\R}{\mathbb{R}}
\newcommand{\Z}{\mathbb{Z}}
%mathcal
\newcommand{\cA}{\mathcal{A}}
\newcommand{\cB}{\mathcal{B}}
\newcommand{\cC}{\mathcal{C}}
\newcommand{\cD}{\mathcal{D}}
\newcommand{\cE}{\mathcal{E}}
\newcommand{\cF}{\mathcal{F}}
\newcommand{\cL}{\mathcal{L}}
\newcommand{\cM}{\mathcal{M}}
\newcommand{\cO}{\mathcal{O}}
\newcommand{\cP}{\mathcal{P}}
\newcommand{\cQ}{\mathcal{Q}}
\newcommand{\cR}{\mathcal{R}}
\newcommand{\cS}{\mathcal{S}}
\newcommand{\cU}{\mathcal{U}}
\newcommand{\cV}{\mathcal{V}}
\newcommand{\cW}{\mathcal{W}}
\newcommand{\cX}{\mathcal{X}}
\newcommand{\cY}{\mathcal{Y}}
\newcommand{\cZ}{\mathcal{Z}}

\newcommand{\hs}{\hspace{0.2in}}
%theorem macros
\newtheorem{thm}{Theorem}
\newtheorem{lem}[thm]{Lemma}
\newtheorem{fact}[thm]{Fact}
\newtheorem{clm}[thm]{Claim}
\newtheorem{rem}[thm]{Remark}
\newtheorem{coro}[thm]{Corollary}
\newtheorem{prop}[thm]{Proposition}
\newtheorem{conj}[thm]{Conjecture}
	\theoremstyle{definition}
\newtheorem{defn}[thm]{Definition}

\theoremstyle{theorem}
\newtheorem{prob}{Problem}


\newcommand{\course}{CS 3000: Algorithms \& Data}
\newcommand{\semester}{Spring 2024}

\newcommand{\hwnum}{1}
\newcommand{\hwdue}{Tuesday January 23 at 11:59pm via Gradescope}

\definecolor{cit}{rgb}{0.05,0.2,0.45} 

\newif\ifsolution

\solutiontrue
%\solutionfalse
\ifsolution
\newcommand{\solution}[1]{\medskip\noindent{\color{DarkBlue}\textbf{Solution:}} #1}
\else
\newcommand{\solution}[1]{}
\fi

\begin{document}
{\Large 
\begin{center} \course\ --- \semester\ \end{center}}
{\large
\vspace{10pt}
\noindent Homework~\hwnum \vspace{2pt}\\
Due~\hwdue}

\vspace{15pt}
\bigskip
{\large
\noindent Name: \yourname \vspace{2pt}\\ Collaborators: \yourcollaborators}

\vspace{15pt}
\begin{itemize}

\item
  Make sure to put your name on the first page.  If you are using the
  \LaTeX~template we provided, then you can make sure it appears by
  filling in the \texttt{yourname} command.

\item This homework is due~\hwdue.  No late assignments will be accepted.  Make sure to submit something before the deadline.

\item Solutions must be typeset.  If you need to draw any diagrams,
  you may draw them by hand as long as they are embedded in the PDF.
  We recommend that you use \LaTeX, in which case it would be best to
  use the source file for this assignment to get started.

\item We encourage you to work with your classmates on the homework
  problems, but also urge you to attempt all of the problems by
  yourself first. \emph{If you do collaborate, you must write all
    solutions by yourself, in your own words.}  Do not submit anything
  you cannot explain.  Please list all your collaborators in your
  solution for each problem by filling in the
  \texttt{yourcollaborators} command.

\item Finding solutions to homework problems on the web, or by asking
  students not enrolled in the class is strictly forbidden.

\end{itemize}
\newpage

%%%%%  PROBLEM 1
\begin{prob}
  \label{prob:mystery}
(4 + 10 = 14 points)  What does this code do?
\end{prob}

\noindent You encounter the following mysterious piece of code.

\begin{algorithm}[H]
\caption{Mystery Function}
\Fn{$F(n)$)}{
  \If{$n=0$}{\Return $(2, 1)$}
  \Else{
    $b \gets 1$\\
    \For{$i$ from 1 to $n$}{
      $b \gets 2b$}
    $(u,v) \gets F(n-1)$\\
    \Return $(u + b, v \cdot b)$}
}
\end{algorithm}

\begin{enumerate}[label=(\alph*)]
\item What are the results of $F(1)$, $F(2)$, $F(3)$, and $F(4)$?  

\solution{
%Put your solution here
$F(1) = (4, 2)$, $F(2) = (8, 8)$, $F(3) = (16, 64)$, $F(4) = (32, 1024)$
}

\item What does the code do in general, when given input integer $n
  \ge 0$? Prove your assertion by induction on $n$.

\solution{
The code gives a coordinate of the form $F(n) = (2^{n + 1}, 2^{\frac{n(n + 1)}{2}})$ when $n \ge 0$.\\\\
\textbf{Inductive Hypothesis}: Let $H(n)$ be the statement $F(n) = (2^{n + 1}, 2^{\frac{n(n + 1)}{2}})$. We will prove that $H(n)$ holds true for $n \ge 0$. \\\\
\textbf{Base Case}: $n = 0$ \\
$F(0) = (2^{0 + 1}, 2^{\frac{0(0 + 1)}{2}})$ \\
$F(0) = (2^{1}, 2^{\frac{0}{2}})$ \\
$F(0) = (2, 1)$ which matches the $n = 0$ if clause of Algorithm 1 so $H(0)$ holds. \\\\
\textbf{Inductive Step}: We will now show that for every $n \ge 0$, $H(n) \rightarrow H(n + 1)$. Assume the inductive hypothesis holds and $H(n) = (2^{n + 1}, 2^{\frac{n(n + 1)}{2}})$. For $n \ge 1$, $b$ is computed by the for loop to be $2^{n}$. That means when we compute $H(n + 1)$, in the \texttt{else-statement}, $b = 2^{n + 1}$ and $(u, v) = H(n) = (2^{n + 1}, 2^{\frac{n(n + 1)}{2}})$. So for the return statement: \\
$H(n + 1) = (u + b, v * b)$ \\
$H(n + 1) = (2^{n + 1} + 2^{n + 1}, 2^{\frac{n(n + 1)}{2}} * 2^{n + 1})$ \\
$H(n + 1) = (2^{1} * 2^{n + 1}, 2^{\frac{n(n + 1)}{2}} * 2^{\frac{2(n + 1)}{2}})$ \\\
$H(n + 1) = (2^{(n + 1) + 1}, 2^{\frac{(n + 2)(n + 1)}{2}})$ \\
$H(n + 1) = (2^{(n + 1) + 1}, 2^{\frac{[(n + 1) + 1](n + 1)}{2}})$ (Inductive hypothesis)
}
\end{enumerate}

\newpage

%%%%% PROBLEM 2
\begin{prob}
  (14 points) Making exact change
\end{prob}

In the country of Perfect Squares, all coins are in denominations that are
perfect squares i.e. $i^2$ for some integer $i$. You need to buy a jacket whose price is an integer $n\ge 1$.
The country is obsessed with being perfect and you can only buy an item if you pay the exact price.
You happen to have exactly one coin of value $i^2$ for each integer $i\ge 2$. 
Additionally you have 4 coins of value $1$ each. Use induction to show that you can pay the exact cost of
the jacket using your coins.

\solution{
There are two cases for the cost: Either the price is a perfect square and you just pay with a coin $i^2$ or the price is not a perfect square which can be thought of as in between two perfect squares. This means that proving any number can be summed using the perfect squares below them and an additional $3$ because there are $4$ $1^{2}$ coins (i.e. $1^{2} + 2^{2} + 3^{2}... + 3$). \\\\
\textbf{Inductive Hypothesis}: For every $n \ge 1$, every integer between $(n - 1)^{2}$ and $n^{2}$ can be composed of coins that compromise of $4$ $1^{2}$ coins and every perfect square value coin. We can call this statement $H(n)$. \\\\
\textbf{Base Case}: Every integer up to $n^{2} = 9$, therefore $n \le 3$, holds. \\
$1 = 1^{2} \\ 2 = 2 * 1^{2} \\ 3 = 3 * 1{2} \\ 4 = 2^{2} \\ 5 = 2^{2} + 1^{2} \\ 6 = 2^{2} + 2 * 1^{2} \\ 7 = 2^{2} + 3 * 1^{2} \\ 8 = 2^{2} + 4 * 1^{2} \\ 9 = 3^{2}$ \\\\
\textbf{Inductive Step}: We will show that after the base case for $n \ge 4$, $H(n) \rightarrow H(n + 1)$. \\
The gap between $(n + 1)^{2}$ and $n^{2}$ is: \\
$(n + 1)^{2} - n^{2} = n^{2} + 2n + 1 - n^{2} = 2n + 1$ \\
Since every sum up to $(n - 1)^{2}$ is assumed to hold, this needs to be greater or equal to the gap: \\
$2n + 1 \le (n - 1)^{2}$ \\
$2n + 1 \le n^{2} - 2n + 1$ \\
$4n \le n^{2}$ \\
$4 \le n$ (Inductive hypothesis) holds consistently after the base case.
}

\newpage

%%%%% PROBLEM 3
\begin{prob}
  (12 points) More induction practice
\end{prob}

Consider the following function $f$ defined on the nonnegative integers.

\begin{align*}
	f(0) & = 3\\
	f(1) & = 4\\
	f(n) & = 3f(n-2) + 2f(n-1)\text{, for $n\ge 2$}
\end{align*}

Prove by induction that $f(n)= (5\cdot (-1)^n + 7\cdot 3^n) / 4$ for all integer $n\ge 0$.

\solution{
\\ \textbf{Inductive Hypothesis}: Let $H(n)$ be the statement $f(n) = \frac{(5 * (-1)^{n} + 7 * 3^{n})}{4}$. We will prove $H(n)$ will hold true for $n \ge 0$. \\\\
\textbf{Base Case}: We will prove $H(0)$ and $H(1)$. \\
$H(0) = \frac{(5 * (-1)^{0} + 7 * 3^{0}}{4})$ \\
$H(0) = \frac{(5 + 7}{4})$ \\
$H(0) = 3$ (Holds true for $f(0)$) \\
$H(1) = \frac{(5 * (-1)^{1} + 7 * 3^{1}}{4})$ \\
$H(1) = \frac{(5 * (-1) + 7 * 3}{4})$ \\
$H(1) = \frac{-5 + 21}{4})$ \\
$H(1) = 4$ (Holds true for $f(1)$) \\\\
\textbf{Inductive Step}: We will now show that for every $n \ge 2$, $H(n) \rightarrow H(n + 1)$. Assume the inductive hypothesis holds and $H(n) = \frac{(5 * (-1)^{n} + 7 * 3^{n})}{4}$. Given the function, $H(n + 1)$ would be: \\
$H(n + 1) = 3H(n - 1) + 2H(n)$ \\
$H(n + 1) = \frac{3(5 * (-1)^{n - 1} + 7 * 3^{n - 1})}{4} + \frac{2(5 * (-1)^{n} + 7 * 3^{n})}{4}$ \\
$H(n + 1) = \frac{(15 * (-1)^{-1} * (-1)^{n} + 7 * 3^{n})}{4} + \frac{(10 * (-1)^{n} + 14 * 3^{n})}{4}$ \\
$H(n + 1) = \frac{(-15 * (-1)^{n} + 7 * 3^{n} + 10 * (-1)^{n} + 14 * 3^{n})}{4}$ \\
$H(n + 1) = \frac{(-5 * (-1)^{n} + 21 * 3^{n})}{4}$ \\
$H(n + 1) = \frac{(5 * (-1)^{n + 1} + 7 * 3^{n + 1})}{4}$ (Inductive hypothesis)
}

\newpage
%%%%%  PROBLEM 4
\begin{prob}
\label{prob:function_growth}
(14 points) Growth of functions
\end{prob}

Arrange the following functions in order from the slowest growing
function to the fastest growing function. Note that $\lg n = \log_2 n$.

\[ n^{2/3} \hs \hs n + \lg n \hs \hs 2^{\sqrt{\lg n}} \hs \hs (\lg n)^{\lg n}\]

Justify your answers.  Specifically, if your order is of the form
\[
f_1 \hs \hs f_2 \hs \hs f_3 \hs \hs f_4,
\]
you should establish $f_1 = O(f_2)$, $f_2 = O(f_3)$, and $f_3 =
O(f_4)$.  For each case, your justification can be in the form of a
proof from first principles or a proof using limits, and can use any
of the facts presented in the lecture or the text.  ({\em Hint:} It
may help to plot the functions and obtain an estimate of their
relative growth rates.  In some cases, it may also help to express the
functions as a power of $2$ and then compare.)

\solution{
$f_1 = 2^{\sqrt{\log_2(n)}}$, $f_2 = n^{\frac{2}{3}}$, $f_3 = n + \log_2(n)$, $f_4 = \log_2(n)^{\log_2(n)}$ \\\\
\textbf{Proof 1}: $f_1 = O(f_2)$ \\
$2^{\sqrt{\log_2(n)}} \le n^{\frac{2}{3}}$ \\
$log_2(2^{\sqrt{\log_2(n)}}) \le log_2(n^{\frac{2}{3}})$ \\
$\sqrt{\log_2(n)} \le \frac{2}{3}log_2(n)$ \\
$\log_2(n) \le \frac{4}{9}(\log_2(n))^{2}$ \\
$\frac{9}{4} \le \log_2(n)$ \\
$2^{\frac{9}{4}} \le n$ for this $n$, $f_1 \le f_2$\\\\
\textbf{Proof 2}: $f_2 = O(f_3)$ \\
for $n \ge 1$, $n^{\frac{2}{3}} \le n$ \\
for $n \ge 1$, $n \le n + \log_2(n)$ \\
$n^{\frac{2}{3}} \le n \le n + \log_2(n)$ \\\\
\textbf{Proof 3}: $f_3 = O(f_4)$ \\
$n + \log_2(n) \le \log_2(n)^{\log_2(n)}$ \\
This can be comparable to if we replace $n = 2^{x}$ \\
$2^{x} + x \le x^{x}$ \\
$2^{x} + x = O(x^{x})$
}

\newpage

%%%%%  PROBLEM 5
\begin{prob}
\label{prob:asymptotics}
(2 $\times$ 7 = 14 points) Properties of asymptotic notation
\end{prob}

\noindent Let $f(n)$, $g(n)$, and $h(n)$ be asymptotically positive
and monotonically increasing functions.  
\begin{itemize}
\item[{\bf (a)}] Using the formal definition of the $O$ and $\Omega$
  notation, prove that if $f(n) = O(h(n))$ and $g(n)^2 = \Omega(h(n)^2)$,
  then $f(n) = O(g(n))$.

\solution{ \\
if $f(n) = O(h(n))$, then $f(n) \le h(n)$ \\
if $g(n)^{2)} = \Omega(h(n)^{2})$, then $g(n)^{2} \ge h(n)^{2}$ \\
Using the first statement, if you square both sides, $f(n)^{2} \le g(n)^{2}$. \\
if $f(n)^{2} \le h(n)^{2}$ and $h(n)^{2} \le g(n)^{2}$, then $f(n)^{2} \le g(n)^{2}$ \\
if you square root both sides, then $f(n) \le g(n)$ \\
This is the same as saying $f(n) = O(g(n))$
}

\item[{\bf (b)}] Give distinct functions $f$ and $g$ satisfying both
  $f(n) = \Theta(g(n))$ and $3^{f(n)} = \Theta(3^{g(n)})$.

  Give distinct functions $f$ and $g$ satisfying $f(n) = O(g(n))$ yet
  $3^{f(n)} \neq O(27^{g(n)})$.

\solution{ \\
$f = n^{2}$ and $g = n^{2} + n$ \\
$\lim_{n \to \infty} \frac{n^{2}}{n^{2} + n} = \frac{1}{1 + \frac{1}{n}} = \frac{1}{1 + 0} = 1$ \\
$f(n) = \Theta(g(n))$ \\
$\lim_{n \to \infty} \frac{3^{n^{2}}}{3^{n^{2} + n}} = \frac{\log_3(3^{n^{2}})}{\log_3(3^{n^{2} + n})} = \frac{n^{2}}{n^{2} + n} = \frac{1}{1 + \frac{1}{n}} = \frac{1}{1 + 0} = 1$ \\
$3^{f(n)} = \Theta(3^{g(n)})$ \\\\
$f = n \ln n$ and $g = n^{2}$ \\
$\lim_{n \to \infty} \frac{n \ln n}{n^{2}} = \frac{\ln n}{n} = \frac{\frac{d}{dn}(\ln n)}{\frac{d}{dn}(n)} = \frac{\frac{1}{n}}{1} = \frac{1}{n} = 0$ \\
$f(n) = O(g(n))$ \\
$\lim_{n \to \infty} \frac{3^{n \ln n}}{3^{3n^{2}}} = \frac{3^{n^{\ln n}}}{3^{3n^{2}}} = \frac{\log_3(3^{n^{\ln n}})}{\log_3(3^{3n^{2}})} = \frac{n^{\ln n}}{3n^{2}} = \frac{\log_n(n^{\ln n})}{\log_n(n^{2^{3}})} = \frac{\ln n}{2^{3}} = \frac{\ln n}{8} = \infty$ \\
$3^{f(n)} \not= O(27^{g(n)})$
}
\end{itemize}


\end{document}
